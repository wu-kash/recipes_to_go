% https://github.com/SvenHarder/xcookybooky/blob/master/xcookybooky.pdf
\begin{recipe}
[% 
portion = {\portion{6}},
preparationtime = {\unit[5]{min}},
bakingtime = {\unit[25]{min}},
]
{Chickpea Shakshuka}
\ingredients[16]{%
\faSquareO \ \textit{Chickpeas} \small{(\weightOz{23})}  \\
\faSquareO \ \textit{Green Olives} \small{(4)}  \\
\faSquareO \ \textit{Olive Oil} \small{(1 tbsp)} \\
\faSquareO \ \textit{Onion} \small{(\sfrac{1}{2} cup)} \\
\faSquareO \ \textit{Red Bell Pepper} \small{(\sfrac{1}{2} cup)} \\
\faSquareO \ \textit{Garlic} \small{(3 cloves)} \\
\faSquareO \ \textit{Crushed Tomato} \small{(\weightOz{28})}  \\
\faSquareO \ \textit{Tomato Paste} \small{(2 tbsp)} \\
\faSquareO \ \textit{Sugar} \small{(2 tsp)} \\
\faSquareO \ \textit{Smoked Paprika} \small{(2 tsp)} \\
\faSquareO \ \textit{Cumin} \small{(1 tsp)} \\
\faSquareO \ \textit{Chill Powder} \small{(2 tsp)} \\
\faSquareO \ \textit{Ground Cinnamon} \small{(\sfrac{1}{4} tsp)} \\
\faSquareO \ \textit{Cayenne Pepper} \small{(1 pinch)} \\
\faSquareO \ \textit{Cardamom} \small{(1 pinch)} \\
\faSquareO \ \textit{Coriander} \small{(1 pinch)} \\
}



\preparation{%
Dice the onion, chop the red bell pepper and mince the garlic.}

\cooking{%
\step{Heat a large rimmed metal or cast iron skillet over medium heat. Once hot, add olive oil, onion, bell pepper and garlic. Saut\'e for 4-5 minutes, stirring frequently, until soft and fragrant.}
\step{Add tomato puree or diced tomatoes, tomato paste, coconut sugar, sea salt, paprika, cumin, chili powder, cinnamon, cayenne pepper (optional), cardamom, and coriander (optional). Stir to combine.}
\step{Bring to a simmer over medium heat and cook for 2-3 minutes, stirring frequently.}
\step{Add chickpeas and olives (optional). Stir to combine. Then reduce heat to medium-low and simmer for 15-20 minutes to allow the flavors to develop and marry with the beans.}
\step{Taste and adjust seasonings as needed, adding more cumin or paprika for smokiness, cayenne for heat, coconut sugar for sweetness, cardamom and coriander for earthiness (or slight curry flavor), chili powder for smoke/heat, or olives for saltiness and to balance the tomato flavor. Cook longer, as needed, to develop flavors.}
\step{Serve as is or with bread, pasta, or rice. I loved this alongside a kale salad, and it went especially well over gluten-free pasta! Garnish with fresh lemon juice, additional olives, and cilantro or parsley for extra flavor (optional).}}

    

    
    
\end{recipe}