% https://github.com/SvenHarder/xcookybooky/blob/master/xcookybooky.pdf
\begin{recipe}
[% 
portion = {\portion{4}},
preparationtime = {\unit[10]{min}},
bakingtime = {\unit[15]{min}},
]
{Chilli Prawns}
    
\ingredients{%
\faSquareO \ \textit{Peeled Prawns} (500 g)  \\
\faSquareO \ \textit{Coconut Milk} (200 ml)  \\
\faSquareO \ \textit{Garlic} (4 cloves)  \\
\faSquareO \ \textit{Ginger Root} (2 cm)  \\
\faSquareO \ \textit{Red Chillies} (2)  \\
\faSquareO \ \textit{Fresh Coriander} (15 g)  \\
\faSquareO \ \textit{Sunflower Oil} (2 tbsp)  \\
\faSquareO \ \textit{Lemon Juice} (1 tbsp)  \\
\faSquareO \ \textit{Tomato Pur\'ee} (1 tbsp)  \\
\faSquareO \ \textit{Garam Masala} (1 tsp)  \\
\faSquareO \ \textit{Salt} (1 tsp)  \\
}


\preparation{%
\\
Peel and crush the garlic and the ginger root. Chop the chilli in half, remove the seeds and cut the chilli into pieces. Wash and cut the coriander,}

\cooking{%
\step Put the oil in a frying pan over moderate heat. Fry the crushed garlic and ginger root for about 3 minutes, keep stirring.
\step Add the prawns and chillies and fry for another 3 minutes.
\step Add the lemon juice, tomato pur\'ee, coconut milk, garam masala and salt. Bring to a boil, then turn down the heat until it is just simmering.
\step Cover and cook for another 5 minutes.
\step Sprinkle the coriander over the chilli prawns before serving.}

    

    
    
\end{recipe}